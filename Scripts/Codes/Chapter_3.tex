% Options for packages loaded elsewhere
\PassOptionsToPackage{unicode}{hyperref}
\PassOptionsToPackage{hyphens}{url}
\PassOptionsToPackage{dvipsnames,svgnames,x11names}{xcolor}
%
\documentclass[
  ignorenonframetext,
]{beamer}
\usepackage{pgfpages}
\setbeamertemplate{caption}[numbered]
\setbeamertemplate{caption label separator}{: }
\setbeamercolor{caption name}{fg=normal text.fg}
\beamertemplatenavigationsymbolsempty
% Prevent slide breaks in the middle of a paragraph
\widowpenalties 1 10000
\raggedbottom
\setbeamertemplate{part page}{
  \centering
  \begin{beamercolorbox}[sep=16pt,center]{part title}
    \usebeamerfont{part title}\insertpart\par
  \end{beamercolorbox}
}
\setbeamertemplate{section page}{
  \centering
  \begin{beamercolorbox}[sep=12pt,center]{part title}
    \usebeamerfont{section title}\insertsection\par
  \end{beamercolorbox}
}
\setbeamertemplate{subsection page}{
  \centering
  \begin{beamercolorbox}[sep=8pt,center]{part title}
    \usebeamerfont{subsection title}\insertsubsection\par
  \end{beamercolorbox}
}
\AtBeginPart{
  \frame{\partpage}
}
\AtBeginSection{
  \ifbibliography
  \else
    \frame{\sectionpage}
  \fi
}
\AtBeginSubsection{
  \frame{\subsectionpage}
}
\usepackage{amsmath,amssymb}
\usepackage{iftex}
\ifPDFTeX
  \usepackage[T1]{fontenc}
  \usepackage[utf8]{inputenc}
  \usepackage{textcomp} % provide euro and other symbols
\else % if luatex or xetex
  \usepackage{unicode-math} % this also loads fontspec
  \defaultfontfeatures{Scale=MatchLowercase}
  \defaultfontfeatures[\rmfamily]{Ligatures=TeX,Scale=1}
\fi
\usepackage{lmodern}
\ifPDFTeX\else
  % xetex/luatex font selection
\fi
% Use upquote if available, for straight quotes in verbatim environments
\IfFileExists{upquote.sty}{\usepackage{upquote}}{}
\IfFileExists{microtype.sty}{% use microtype if available
  \usepackage[]{microtype}
  \UseMicrotypeSet[protrusion]{basicmath} % disable protrusion for tt fonts
}{}
\makeatletter
\@ifundefined{KOMAClassName}{% if non-KOMA class
  \IfFileExists{parskip.sty}{%
    \usepackage{parskip}
  }{% else
    \setlength{\parindent}{0pt}
    \setlength{\parskip}{6pt plus 2pt minus 1pt}}
}{% if KOMA class
  \KOMAoptions{parskip=half}}
\makeatother
\usepackage{xcolor}
\newif\ifbibliography
\usepackage{graphicx}
\makeatletter
\def\maxwidth{\ifdim\Gin@nat@width>\linewidth\linewidth\else\Gin@nat@width\fi}
\def\maxheight{\ifdim\Gin@nat@height>\textheight\textheight\else\Gin@nat@height\fi}
\makeatother
% Scale images if necessary, so that they will not overflow the page
% margins by default, and it is still possible to overwrite the defaults
% using explicit options in \includegraphics[width, height, ...]{}
\setkeys{Gin}{width=\maxwidth,height=\maxheight,keepaspectratio}
% Set default figure placement to htbp
\makeatletter
\def\fps@figure{htbp}
\makeatother
\setlength{\emergencystretch}{3em} % prevent overfull lines
\providecommand{\tightlist}{%
  \setlength{\itemsep}{0pt}\setlength{\parskip}{0pt}}
\setcounter{secnumdepth}{-\maxdimen} % remove section numbering

\usepackage{textpos}
\setbeamertemplate{headline}{
  \begin{textblock*}{5cm}(10.2cm,0.2cm)
  \includegraphics[width=2.2cm]{UniUrb-logo.png}
  \end{textblock*}}
  
  \definecolor{myblue}{HTML}{005997}
  \setbeamercolor{frametitle}{fg=myblue}
    \setbeamercolor{framesubtitle}{fg=myblue}
    \setbeamercolor{frametitle right}{fg=myblue}
  \setbeamercolor{titlelike}{fg=myblue}
    \setbeamercolor{title}{fg=myblue}
      \setbeamercolor{subtitle}{fg=myblue}
    \setbeamercolor{part title}{fg=myblue}
    \setbeamercolor{section title}{fg=myblue}
    \setbeamercolor{subsection title}{fg=myblue}
  \setbeamercolor{section name}{fg=myblue}
  \setbeamercolor{subsection name}{fg=myblue}
  \setbeamercolor{part name}{fg=myblue}
  \setbeamercolor{title in head/foot}{fg=myblue}
  \setbeamercolor{subtitle in head/foot}{fg=myblue}
  \setbeamercolor{block title}{fg=myblue}
  
  \setbeamercolor{bullet}{fg=myblue}
  \setbeamercolor{section in toc}{fg=myblue}
  \setbeamercolor{subsection in toc}{fg=myblue}
  \setbeamercolor{section in head/foot}{fg=myblue}
  \setbeamercolor{subsection in head/foot}{fg=myblue}
  
  
  \setbeamercolor{itemize item}{fg = myblue}
  \setbeamercolor{itemize subitem}{fg = myblue}
  \setbeamercolor{itemize subsubitem}{fg = myblue}
  \setbeamercolor{enumerate item}{fg = myblue}
  \setbeamercolor{enumerate subitem}{fg = myblue}
  \setbeamercolor{enumerate subsubitem}{fg = myblue}
  
 
  
\usepackage{etoolbox}
\AtBeginEnvironment{thebibliography}{\scriptsize}
\ifLuaTeX
  \usepackage{selnolig}  % disable illegal ligatures
\fi
\usepackage[round]{natbib}
\bibliographystyle{plainnat}
\usepackage{bookmark}
\IfFileExists{xurl.sty}{\usepackage{xurl}}{} % add URL line breaks if available
\urlstyle{same}
\hypersetup{
  pdftitle={Mapping the Evolution of the Digital Divide: Comparative Analysis of European Research and the Multifaceted Dimensions in Italian Enterprises},
  pdfauthor={Luis Carlos Castillo},
  colorlinks=true,
  linkcolor={myblue},
  filecolor={Maroon},
  citecolor={Blue},
  urlcolor={Blue},
  pdfcreator={LaTeX via pandoc}}

\title{Mapping the Evolution of the Digital Divide: Comparative Analysis
of European Research and the Multifaceted Dimensions in Italian
Enterprises}
\author{Luis Carlos Castillo}
\date{24 June 2024}
\institute{Binational Doctorate\\
University of Urbino and Universty of Bremen\\
Ph.D.~Program in Global Studies\\
Supervisor: Prof.~Francesco Vidoli\\
Co-Supervisor: Prof.~Dr.~Christian Cordes}

\begin{document}
\frame{\titlepage}

\begin{frame}{Cumulative Dissertation}
\phantomsection\label{cumulative-dissertation}
\begin{center}
\includegraphics[width=1\textwidth, height=0.75\textheight]{structure_thesis.png}
\end{center}
\end{frame}

\begin{frame}{Chapter 3}
\phantomsection\label{chapter-3}
\begin{block}{\centering{Exploring Patterns and Shifts: Factor Analysis of the First  and Second-Level Digital Divide in Italian Enterprises}}
\phantomsection\label{section}
\end{block}
\end{frame}

\begin{frame}{Content I}
\phantomsection\label{content-i}
\begin{enumerate}
\item
  Introduction
\item
  Digital divide overview
\item
  Objectives and Research Questions
\item
  Theoretical Framework
\item
  Data
\item
  Methodology
\item
  Results
\end{enumerate}
\end{frame}

\begin{frame}{1. Introduction}
\phantomsection\label{introduction}
\begin{itemize}
\item
  DT play a pivotal role in reshaping business models
  \citep{trischler2023} , driving innovation \citep{ciarli2021}, and
  fostering competitive advantages \citep{jeganjosephjerome2024}.
\item
  The unequal adoption of DT has led to significant challenge: the
  digital divide.
\item
  The digital divide at the firm level is a multifaceted gap
  characterized not only by disparities in access, skills and usage of
  DT, but also the derived benefits from different types of use.
\item
  While the digital divide is a global issue, its impact on enterprises,
  especially in Italy, presents unique challenges and opportunities.
\end{itemize}
\end{frame}

\begin{frame}{2. Digital divide overview}
\phantomsection\label{digital-divide-overview}
\begin{itemize}
\item
  The digital divide at the firm level is a topic that needs further
  exploration
  \citep{pejicbach2013, acilar2021, lythreatis2022, thonipara2023, shakina2021a, siqueira2019}.
\item
  \textbf{Fragmented research:} Highlights the different frameworks and
  measurements in the field.
  \citep{forman2002, vehovar2006, shakina2021a, acilar2021, ndulu2022}.
\item
  \textbf{Impact on Business Performance:} Addresses how DT impact
  innovation \citep{usai2021}, productivity \citep{cusolito2020}, and
  wages between high and low skilled labor \citep{vasilescu2020} .
  Leaving room for further analysis on the underlying causes of digital
  disparities.
\end{itemize}
\end{frame}

\begin{frame}{3. Objectives}
\phantomsection\label{objectives}
\begin{itemize}
\item
  Develop composite indices to track trends in the first and
  second-level digital divide.
\item
  Analyze variations in the first and second-level digital divide across
  firm sizes, sectors, and regions in Italy.
\item
  Evaluate the alignment of resources and technology integration theory
  with observed data.
\item
  Propose targeted policy interventions based on research findings.
\end{itemize}
\end{frame}

\begin{frame}{3.1. Research Questions}
\phantomsection\label{research-questions}
\begin{itemize}
\item
  How does the Resources and Technology Integration theory reflect
  observed patterns in DT adoption across different Italian enterprises?
\item
  What are the evolving patterns and shifts in access, skills, and usage
  among Italian enterprises, and how do these dynamics reflect the
  disparities between the three indices across different firm sizes,
  sectors, and regions over a six-year period?
\item
  According to the findings, what policy interventions can be
  recommended to help Italian enterprises more effectively bridge the
  digital divide?
\end{itemize}
\end{frame}

\begin{frame}{4. Theoretical Framework}
\phantomsection\label{theoretical-framework}
\begin{itemize}
\tightlist
\item
  Adaptation of the Resources and Appropriation Theory (van Dijk 2005,

  \begin{enumerate}
  [1)]
  \setcounter{enumi}{2019}
  \tightlist
  \item
    toward the Resources and Technology Integration Theory.
  \end{enumerate}
\end{itemize}

\begin{center}
\includegraphics[width=1\textwidth, height=0.75\textheight]{dd_model.png}
\end{center}
\end{frame}

\begin{frame}{5. Data}
\phantomsection\label{data}
\begin{itemize}
\item
  The dataset was derived from the ICT Usage in Enterprises Survey
  conducted annually between 2014 and 2019 by ISTAT.
\item
  Each year, the survey collected an average of 19741.5 observations.
\item
  The analysis focused on a selection of variables from the survey that
  met certain criteria.
\item
  Considerations and Limitations.

  \begin{itemize}
  \tightlist
  \item
    Independence of Annual Data
  \end{itemize}
\item
  Data treatments, codes, and summary statistics are available on my
  \href{https://github.com/luchocastillo84/Factor_Analysis_Digital_Divide/tree/main}{GitHub
  repository} for replicability and further analysis.
\end{itemize}
\end{frame}

\begin{frame}{5.1 List of Variables I}
\phantomsection\label{list-of-variables-i}
\begin{itemize}
\item
  In total 26 variables were used.

  \begin{itemize}
  \tightlist
  \item
    Access index 6 variables
  \item
    Skills index 9 variables
  \item
    Usage index 11 variables
  \end{itemize}
\end{itemize}

\begin{center} 
\includegraphics[width=1\textwidth]{access_vars.png}
\end{center}

\begin{itemize}
\tightlist
\item
  \begin{enumerate}
  [(A)]
  \setcounter{enumi}{2}
  \tightlist
  \item
    Continuous variables, (B) Binary variables.
  \end{enumerate}
\end{itemize}
\end{frame}

\begin{frame}{5.2 List of Variables II}
\phantomsection\label{list-of-variables-ii}
\begin{center}
\includegraphics[width=1\textwidth,height=0.7\textheight]{skills_usage_vars.png}
\end{center}
\end{frame}

\begin{frame}{6 Methodology}
\phantomsection\label{methodology}
\begin{itemize}
\tightlist
\item
  The indices were constructed using dimensionality reduction techniques
  Factor Analysis for Mixed Data (FAMD) and Multiple Correspondence
  Analysis (MCA).
\item
  Visualise and measure adoption patterns.
\end{itemize}

\[ \text{Access Index}_i = a_1 \cdot X1_i + a_2 \cdot X2_i + \ldots + a_6 \cdot X6_i \]
\[ \text{Skills Index}_i = b_1 \cdot Y1_i + b_2 \cdot Y2_i + \ldots + b_6 \cdot Y6_i \]
\[ \text{Usage Index}_i = c_1 \cdot Z1_i + c_2 \cdot Z2_i + \ldots + c_{11} \cdot Z11_i \]

Where \(\mathbf{X}\), \(\mathbf{Y}\), and \(\mathbf{Z}\) are the
matrices of observations, and \(\mathbf{a}\), \(\mathbf{b}\), and
\(\mathbf{c}\) are the vectors of weights derived from FAMD and MCA
contributions of each variable to the retained dimensions.
\end{frame}

\begin{frame}{7. Results I}
\phantomsection\label{results-i}
\includegraphics{Chapter_3_files/figure-beamer/yearTrendSize-1.pdf}
\end{frame}

\begin{frame}{7. Key takeaways I}
\phantomsection\label{key-takeaways-i}
\begin{itemize}
\item
  While SMEs can maintain continuous growth in access, larger firms face
  different challenges.
\item
  These findings align with Bratta et al.~(2020), who discuss the
  hyper-depreciation measure issued in 2016.
\item
  SMEs face more challenges in acquiring digital skills. There is a
  significant shortage of ICT graduates in Italy according to the
  ``Digital Skills Observatory'' in 2019.
\item
  A higher usage index is present in SMEs as existing technologies need
  to be operated either by outsourcing digital skills or maximising the
  existing workforce.
\end{itemize}
\end{frame}

\begin{frame}{7.1. Results II}
\phantomsection\label{results-ii}
\includegraphics{Chapter_3_files/figure-beamer/yearTrendMac-1.pdf}
\end{frame}

\begin{frame}{7.1. Key takeaways II}
\phantomsection\label{key-takeaways-ii}
\begin{itemize}
\item
  Notable progress in digital access across sectors due to effective
  policy measures like hyper-depreciation.
\item
  The digital skills index remains low, highlighting challenges in
  developing necessary competencies.
\item
  Commercial, Industrial, Service Sectors: Growing digital access and
  stable usage, but significant skills gap.
\item
  Creative Industries and ICT Sector: High levels of access and skills
  but moderate usage, reflecting different technological needs in
  workflows.
\end{itemize}
\end{frame}

\begin{frame}{7.2. Results III}
\phantomsection\label{results-iii}
\includegraphics{Chapter_3_files/figure-beamer/yearTrendReg-1.pdf}
\end{frame}

\begin{frame}{7.2. Key takeaways III}
\phantomsection\label{key-takeaways-iii}
\begin{itemize}
\item
  Northern and Central Regions:

  \begin{itemize}
  \tightlist
  \item
    Significant improvements in digital access.
  \item
    Benefits from robust industrial bases and policy interventions.
  \item
    Persistent skills gap, slight increase towards the end.
  \end{itemize}
\item
  South and Insular Regions:

  \begin{itemize}
  \tightlist
  \item
    Lag behind in both access and skills. Reflecting systemic issues.
  \item
    Need targeted policy efforts.
  \end{itemize}
\item
  Usage Trends:

  \begin{itemize}
  \tightlist
  \item
    Similar trends across regions.
  \item
    DT integration driven by national policies, market forces, and
    sectoral requirements.
  \end{itemize}
\end{itemize}
\end{frame}

\begin{frame}{7..3 Evaluating Theory in Light of Data}
\phantomsection\label{evaluating-theory-in-light-of-data}
Sequential Model Challenges: While the theory posits a sequential model
of technology adoption (access → skills → usage), our data shows that
skills development has not kept pace with access, and usage sometimes
surpasses skills. This indicates that firms may adopt technologies
without fully developing digital skills, relying on external support or
simpler applications.

Inequality in Resources: The theory suggests inequality in access to
resources leads to disparities in technology adoption. However, our data
only infers this inequality without direct proof.
\end{frame}

\begin{frame}{7.4. Key Interventions}
\phantomsection\label{key-interventions}
Accelerate Digital Skills Acquisition: Promote and subsidize e-learning
platforms and certification programs for rapid, flexible digital skills
training.

Enhance Digital Infrastructure and Support for SMEs in Lagging Regions:
Expand broadband and mobile network coverage in under-served regions
through public-private partnerships.
\end{frame}

\begin{frame}{8. Q\&A}
\phantomsection\label{qa}
Thank you for your attention
\end{frame}

\renewcommand\refname{References}
\begin{frame}[allowframebreaks]{References}
  \bibliographytrue
  \bibliography{ShiftPatterns.bib}
\end{frame}

\end{document}
